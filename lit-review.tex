\chapter*{Обзор литературы}
\addcontentsline{toc}{chapter}{Обзор литературы}  
В настоящее время задача классификации хорошо изучена.
Для нее существует множество обобщений: в \cite{binclass-generalization-3} рассказывается о нескольких способах обобщения бинарной классификации на мультиклассовую, в \cite{decision-trees-original} предлагают решение задачи классификации на основе решающих деревьев, в \cite{binclass-generalization-1,lee_N-1} предлагают улучшение решающих дереьвев на основе \gls{svm}, описанных в \cite{svm_original}.
Сама же идея популярного алгоритма Random Forest, содержащим уже упомянутые деревья решений, описана в \cite{cite:random-forest}.
Подходов к решению задач мультиклассовой классификации действительно много, и статья \cite{multi_class_review} дает обзор наиболее популярных методов с ссылками на соответствующие работы.
Большинство из этих подходов также описано в уже ставшем классическим учебнике \cite{bishop}, откуда были подчерпнуты идеи для построения первого прототипа модели классификатора.
О типовом способе векторизации текста \acrshort{tf-idf} и способов его интерпретации подробно рассказано в \cite{cite:tf-idf-interpretation}.

Недавная статья Google \cite{cite:ml-debt} стала отправной точкой для определения технического долга в \acrshort{ml}-системах.
Авторы приводят подробное описание основных проблем и намечают пути решения, однако никаких конкретных алгоритмов, равно как и инструментов, не приводят.
В статье \cite{cite:ml-reproducibility} авторы приводят свое видение стандартов воспроизводимости экспериментов и процессов в области \acrshort{ml}.
Этот документ подробно описывает каждый пункт, который авторы включили в описываемой ими работе, а также намечает основные проблемы в достижении каждой ступени стандарта.
Описание проблем и стандартов воспроизводимости стало ориентиром в настоящей работе, но отсутствие конкретных рекомендаций по реализации привело к дальнейшему поиску.

Одним из решений для воспроизводимости является фреймворк \gls{dvc}, описанный в \cite{cite:dvc}.
Он покрывает большинство описанных в \cite{cite:ml-debt} проблем и удовлетворяет стандартам из \cite{cite:ml-reproducibility}.
Тем не менее, готовых практических примеров и хороших практик работы с этим инструментом в контексте поставленной задачи классификации документов на момент написания работы не найдено.
Опираясь на это, в настоящей работе была сформирована задача разработки процесса автоматизации и унификации обучения существующего классификатора с последующим расширением на новые локализации.
