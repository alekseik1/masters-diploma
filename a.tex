%!TEX root = thesis.tex"`

% \chapter{Пример расчета\\ термодинамических свойств \texorpdfstring{\acrshort{ifg}}{ИФГ}}
% Модуль легко ставится через \Gls{pypi}:
% \begin{minted}{bash}
% $ pip install numpy
% $ pip install ifg
% \end{minted}

% Основной функционал содержится в классе \texttt{IfgCalculator}.
% \begin{minted}{Python}
% from ifg import IfgCalculator
% import numpy as np
% specific_volumes = [0.33, 1., 10.,]
% temperatures = np.linspace(1e-4, 1e4, 10**6)
% calculator = IfgCalculator(
%     temperatures=temperatures,
%     specific_volumes=specific_volumes,
%     input_in_si=False,  # False даст в атомной системе
%     output_in_si=False
% )
% \end{minted}
% После этого все свойства будут доступны как поля экземпляра \texttt{calculator}:
% \begin{minted}{Python}
% >>> calculator.p
% array([[1.21465823e+01, 1.91415600e+00, 4.12392425e-02],
%        [1.21466330e+01, 1.91419106e+00, 4.12555143e-02],
%        [1.21467832e+01, 1.91429487e+00, 4.13036646e-02],
%        ...,
%        [3.03030975e+04, 9.99999392e+03, 9.99998139e+02],
%        [3.03031278e+04, 1.00000039e+04, 9.99999139e+02],
%        [3.03031581e+04, 1.00000139e+04, 1.00000014e+03]])
% >>> calculator.C_P
% array([[4.92449475e-05, 1.03122265e-04, 4.78651124e-04],
%        [4.97374654e-03, 1.04153735e-02, 4.83453163e-02],
%        [9.89825789e-03, 2.07277142e-02, 9.62203561e-02],
%        ...,
%        [2.49998418e+00, 2.49999478e+00, 2.49999948e+00],
%        [2.49998418e+00, 2.49999478e+00, 2.49999948e+00],
%        [2.49998418e+00, 2.49999478e+00, 2.49999948e+00]])
% \end{minted}
% Во втором примере видим, как теплоемкость $C_P$ стремится к $5/2$ при высоких температурах ---~ ожидаемое поведение для \acrshort{ifg}.

% Можно также проверить, что давление стремится к конечному пределу при низких температурах.
% Изменим входные данные:
% \begin{minted}{Python}
% temperatures = np.linspace(1e-10, 1e-8, 10**4)
% calculator = IfgCalculator(
%     temperatures=temperatures,
%     specific_volumes=specific_volumes,
%     input_in_si=False,
%     output_in_si=False
% )
% \end{minted}
% А затем запустим расчет давления:
% \begin{minted}{Python}
% >>> calculator.p
% array([[12.14658227,  1.914156  ,  0.04123924],
%        [12.14658227,  1.914156  ,  0.04123924],
%        [12.14658227,  1.914156  ,  0.04123924],
%        ...,
%        [12.14658227,  1.914156  ,  0.04123924],
%        [12.14658227,  1.914156  ,  0.04123924],
%        [12.14658227,  1.914156  ,  0.04123924]])
% \end{minted}
% Получены конечные пределы.

% Больше примеров можно найти по ссылке \url{https://ifg-py.readthedocs.io/en/latest/class_desc.html}.
% Код графиков \ref{fig:chemical_potential}, \ref{fig:heat_capacity}, \ref{fig:sound_velocity} доступен в репозитории: \url{https://github.com/alekseik1/ifg-py/tree/master/examples/plots}.
