%!TEX root = thesis.tex"`

\intro

\textbf{Выбор и актуальность темы.} 
Задача классификации возникает повсеместно в области анализа данных.
Определение наличия лица на фотографиии, отнесение письма к категории спама, разделение документов по их категории -- все перечисленное задачи можно рассматривать как проблему классификации.
На момент написания работы существует множество подходов к решению подобного рода задач, рассматривающих проблемы из области Data Science.
Тем не менее, при работе в промышленных масштабах недостатчно одной лишь модели: необходимо также выстраивание вокруг нее инфраструктуры, обеспечивающей надежность решения.
Такая инфраструктура включает в себя непрерывное функционирование, ведение журналов действия, развертывание обновлений, наблюдаемость состояния и прочие компоненты, используемые в \gls{swe}.
Зачастую к таким системам дополнительно предъявляется требование автономности: ожидается, что участие человека в отлаженном процессе будет минимальным, а любое улучшение системы возможно осуществить в автоматическом режиме.

Построение такого рода автоматизированных систем классификации на сегодняшний день является сложной задачей, требующей решения как теоретических проблем из области анализа данных, так и практических задач из области программной инженерии.
Вариативность в выборе моделей, сложности при выстроении повторяемого процесса обучения и различия в предметной специфики областей Data Science и \gls{swe} зачастую приводят к тому, что получающаяся система либо остается на этапе прототипа, сосредоточенного на свойствах модели и особенностях данных, либо же погружается в технические детали реализации, не оставляя места для будущего экспериментирования с моделью и снижая темпы улучшения качества классификации.
Поиску сбалансированного решения, использующего лучшие практики \gls{swe} и сохраняющего простоту в улучшении \acrshort{ml}-части, посвящено много работ (как отдельных исследователей \cite{cite:ml-reproducibility}, так и крупных компаний \cite{cite:ml-debt}).

\textbf{Цели и задачи работы.}
Основной целью настоящей работы является разработка классификатора документов, поддерживающего работу с несколькими языками, документами в разнообразных форматах и способного самостоятельно дообучаться на новых данных и дополнительной информации об ошибках на предыдущей итерации.

Для достижения этой цели были поставлены и решены следующие задачи:
\begin{enumerate}
    \item Автоматизация существующего прототипа модели; разработка единого пайплайна для обучения, способного к расширению данных.
    \item Увеличение списка поддерживаемых локализаций с использованием универсального пайплайна обучения.
    \item Предоставление и обеспечение хранения артефактов моделей для каждой локализации.
    \item Выстраивание процессов разработки с учетом имеющихся практик командной работы в \gls{swe}.
    \item Поиск и устранение ошибок, возникающих при автоматизации и обобщении пайплайна обучения на новых языках.
    \item Привлечение новых наборов данных для обучения классификатора на других языках.
\end{enumerate}

\textbf{Научная значимость работы.}
\begin{enumerate}
    \item Впервые был разработан алгоритм автоматизации процесса обучения классификатора документов на основе \gls{dvc} и шаблонов.
    \item Впервые осуществлено обобщение пайплайна обучения с одного языка на несколько других.
\end{enumerate}

\textbf{Практическая значимость работы.}
\begin{enumerate}
    \item С помощью универсального \gls{pipeline} обучения стало возможным применять существующую модель на других языках.
    \item Благодаря выстроению автоматического контроля в процессе обучения стало возможным ускорить работу других разработчиков и вести параллельную разработку нескольких функциональностей.
\end{enumerate}
