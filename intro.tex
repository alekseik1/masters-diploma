%!TEX root = thesis.tex"`

\intro

% Свойства электронного газа необходимо знать для решения широкого круга задач астрофизики, физики твердого тела, физики плазмы, химии и молекулярной биологии; кроме того, модель электронного газа играет важную роль в статистической физике. Учет квантовых свойств электронов позволил успешно объяснить такие явления, как эволюция и коллапс звезд, взрыв новых и сверхновых, проводимость полупроводников и металлов, сверхпроводимость и многие другие. Широкую известность получила модель невзаимодействующего (идеального) ферми-газа в силу своей простоты. Между тем, учет взаимодействия для системы квантовых частиц представляет собой чрезвычайно сложную задачу статистической физики, не решенную до настоящего времени. Особые сложности представляет кулоновское взаимодействие, так как в этом случае необходимо учитывать дальнодействующий характер потенциала. Поэтому особое значение приобретают модельные системы, на примере которых можно предсказывать свойства реальных веществ. Наиболее популярной моделью такого рода является модель <<желе>>~--- электронов на однородном несжимаемом компенсирующем положительном фоне. Эта модель позволяет в чистом виде вычислить обменно--корреляционную энергию, которая необходима для построения так называемых обменно--корреляционных функционалов для метода функционала плотности (\acrshort{mfp}). В настоящее время \acrshort{mfp} очень активно применяется для моделирования различных свойств реальных веществ и играет большое значение в квантовой химии, физике твердого тела и физике неидеальной плазмы. Модель <<желе>> хорошо изучена при нулевой температуре, однако исследования этой модели при конечных температурах активно продолжаются и в наше время, и на сегодняшний день далеки от завершения. Результатом изучения модели <<желе>> при конечной температуре должно стать создание надежных обменно--корреляционных функционалов с явной зависимостью от температуры. Большой интерес представляют также и другие свойства модели <<желе>>, в частности, диэлектрическая функция. Поэтому \underline{весьма актуальной} является программная реализация метода функционала плотности для взаимодействующего электронного газа, позволяющая, помимо термодинамических свойств, рассчитать также транспортные и оптические свойства.  

% \underline{Цель работы}~--- получение аналитических выражений для термодинамических функций и термодинамических коэффициентов модели идеального ферми--газа, а также разработка алгоритма программной реализации метода функционала плотности для взаимодействующего электронного газа на однородном компенсирующем фоне положительного заряда.
