%!TEX root = thesis.tex"`

\conclusion

В даной работе получены следующие результаты:

\begin{enumerate}
    \item Прототип модели классификации документов для одной локализации выведен в продуктив, на его основе разработан \gls{pipeline} обучения от начала до конца.
    \item Модель обобщена на другие локализации (см. таблицу \ref{table:supported_locales}) с единый пайплайном.
    \item Выстроен процесс разработки, настроена система ведения журналов и поиска неисправностей кода.
    \item Автоматизирован процесс дообучения модели на новых данных.
    % \item Получены аналитические выражения для вторых производных термодинамического потенциала и термодинамических коэффициентов \acrshort{ifg}.
    % \item Получены асимптотические выражения для всех термодинамических функций \acrshort{ifg} в пределе низких и высоких температур.
    % \item Разработана общедоступная программная реализация для модели \acrshort{ifg} на языке \Gls{python}.
    % \item Получен гамильтониан изотропной модели \acrshort{veg}.
    % \item Разработан алгоритм программной реализации \acrshort{mfp} для модели \acrshort{veg}.
\end{enumerate}

В дальнейшем планируется расширять список поддерживаемых локализаций, добавлять систему контроля качества и автоматизировать ее посредством \gls{cicd}.

\begin{table}[h]
    \centering
    \begin{tabular}{|c|c|c|}
        \hline
                 & precision & recall  \\ \hline
        German   & 0.98520   & 0.97661 \\ \hline
        English  & 0.95502   & 0.95358 \\ \hline
        Spanish  & 0.99263   & 0.99159 \\ \hline
        French   & 0.97535   & 0.97113 \\ \hline
        Italian  & 0.97212   & 0.95854 \\ \hline
        Japanese & 0.95297   & 0.91297 \\ \hline
        Russian  & 0.99175   & 0.99071 \\ \hline
    \end{tabular}
    \caption{Таблица поддерживаемых локализаций}
    \label{table:supported_locales}
\end{table}
