%!TEX root = thesis.tex"`
\usepackage[acronym,toc]{glossaries}
\makenoidxglossaries

\newacronym{mcc}{MCC}{Multiclass classification}
\newacronym{sshfs}{SSHFS}{SSH File System}
\newacronym{ovr}{OvR}{One-vs.-rest}
\newacronym{vcs}{VCS}{Version Control System}
\newacronym{dag}{DAG}{Directed Acyclic Graph}
\newacronym{nlp}{NLP}{Natural Language Processing}
\newacronym{gb}{Gb}{Gigabyte, 1024 Megabytes}
\newacronym{ovo}{OvO}{One-vs.-one}
\newacronym{tf}{TF}{Term Frequency, частота слова}
\newacronym{idf}{IDF}{Inverse Document Frequency, обратная частота документа}
\newacronym{ml}{ML}{Машинное обучение}
\newacronym{dbms}{DBMS}{Database Management System, также СУБД -- система управления базами данных}
\newacronym{tf-idf}{TF-IDF}{Term frequency - Inverse Document Frequency}
\newglossaryentry{sup_learning}
{
    name=supervised learning,
    description={Постановка задачи, при которой существует т.н. \textit{обучающая выборка}, в которой у каждого элемента известна целевая переменная}
}

\newglossaryentry{semisup_learning}
{
    name=semi-supervised learning,
    description={Постановка задачи, при которой существует выборка, состоящая из $l$ элементов с известной разметкой и $u$ элементов с неизвестной разметкой}
}
\newglossaryentry{unsup_learning}
{
    name=unsupervised learning,
    description={Постановка задачи, при которой не существует выборки с заранее известной целевой переменной}
}
\newglossaryentry{binary_classification}
{
    name=binary classification,
    description={Задача классификации с $\mathbb{Y} = \{0, 1\}$}
}
\newglossaryentry{linear_classifier}
{
    name=linear classifier,
    description={Классификатор вида $y = f\left( \sum\limits_{j}^{} w_j x_j \right)$}
}
\newglossaryentry{svm}
{
    name=support vector machine,
    description={Метод классификации, основанный на пострении разделяющей гиперплоскости так, чтобы отступ от плоскости до объектов разных классов был максимален}
}
\newglossaryentry{soft_margin}
{
    name=soft margin,
    description={Метод классификации, основанный на пострении разделяющей гиперплоскости так, чтобы отступ от плоскости до объектов разных классов был максимален}
}
\newglossaryentry{kernel}
{
    name=kernel,
    description={Функция вида $K(x, y) = \vec{\phi(\vec{x})} \cdot \vec{\phi(\vec{y})}$, использующаяся для улучшения качества на нелинейных зависимостях между входными данными и целевой переменной}
}
\newglossaryentry{hyperparameters}
{
    name=hyperparameters,
    description={Набор численных параметров, которые нельзя оптимизировать на основе обучающей выборки}
}
\newglossaryentry{decision-tree}
{
    name=decision tree,
    description={Алгоритм, основанный на построении дерева, каждый узел которого разбивает данные в соответствии с некоторым условием (чаще всего разбивается по пороговой функции)}
}
\newglossaryentry{binary-decision-tree}
{
    name=binary decision tree,
    description={Решающее дерево, у которого каждая вершина имеет не более двух потомков}
}
\newglossaryentry{pipeline}
{
    name=pipeline,
    description={Последовательность действий в некотором сложном процессе (обработка данных, бизнес-процесс команды и т.п.)}
}
\newglossaryentry{random-forest}
{
    name=random forest,
    description={Ансамбль отдельно взятых решающих деревьев (\gls{decision-tree}), построенных с использованием беггинга и случайных подпространств и дающий высокое качество решения в задачах классификации}
}
\newglossaryentry{crawler}
{
    name=crawler,
    description={Программа для обхода веб-страниц и выгрузки ее контента для дальнейшей обработки (индексация страниц, поиск документов, выгрузка текстов и т.д.)}
}
\newglossaryentry{golang}
{
    name=golang,
    description={Компилируемый многопоточный язык общего назначения со строгой типизацией. \href{https://go.dev/}{Официальная страница}}
}
\newglossaryentry{sqlite}
{
    name=SQLite,
    description={Минималистичная \acrshort{dbms}, предназначенная для быстрого создания легковесного хранлища (с возможность хранения в единственном файле), \href{https://www.sqlite.org/index.html}{официальная страница}}
}
\newglossaryentry{apache}
{
    name=Apache,
    description={Фонд программного обеспечения, поддерживаемый крупными IT-компаниями и множеством независимых разработчиков со всего мира}
}
\newglossaryentry{tika}
{
    name=Tika,
    description={Программа для определения и выделения текста и метаинформации из текста, предствленного в разнообразных форматах, \href{https://tika.apache.org/}{официальная страница}}
}
\newglossaryentry{lemmatization}
{
    name=lemmatization,
    description={Процесс приведения слова к нормальной форме}
}
\newglossaryentry{spacy}
{
    name=spaCy,
    description={Фреймворк для \acrshort{nlp}, \href{https://spacy.io/}{официальный сайт}}
}
\newglossaryentry{nltk}
{
    name=NLTK,
    description={Natural Language Toolkit, фреймворк для \acrshort{nlp}, \href{https://www.nltk.org/}{официальный сайт}}
}\newglossaryentry{scikit-learn}
{
    name=scikit-learn,
    description={Библиотека с инструментами машинного обучения для \gls{python}}
}
\newglossaryentry{python}
{
    name=Python,
    description={Интерпретируемый язык программирования с динамической типизацией, популярный в области Data Science}
}
\newglossaryentry{pca}
{
    name=PCA,
    description={Principal Component Analysis, метод главных компонент -- способ уменьшения размерности данных без существенной потери информации}
}
\newglossaryentry{swe}
{
    name=Software Engineering,
    description={Дисциплина информатики, изучающая процесс создания программных продуктов с технической точки зрения}
}
\newglossaryentry{devops}
{
    name=DevOps,
    description={сокр. от \textbf{Dev}elopment and \textbf{Op}eration\textbf{s} -- набор практик для ускорения цикла разработки и упрощения каждой стадии данного цикла}
}
\newglossaryentry{vps}
{
    name=VPS,
    description={Virual Private Server, арендуемый сервер с заранее заданной мощностью}
}
\newglossaryentry{s3}
{
    name=S3,
    description={сокр. от Simple Storage Service -- изобретенный в \href{https://aws.amazon.com}{Amazon} способ хранения объектов, впоследствии ставший популярным среди облачных провайдеров}
}
\newglossaryentry{dvc}
{
    name=DVC,
    description={сокр. от Data Version Control -- инструмент для версионировния данных и построения \gls{pipeline} в машинном обучении. \href{https://dvc.org/}{Официальная страница}}
}
\newglossaryentry{jupyter-notebook}
{
    name=Jupyter Notebook,
    description={Интерактивная среда для исполнения команд (преимущественно на \gls{python}), поддерживающая встраиваемую визуализацию, разметку на нескольких языках и создание отчетов}
}
\newglossaryentry{git}
{
    name=git,
    description={Популярная среди разработчиков ПО \acrshort{vcs}}
}
\newglossaryentry{hadoop}
{
    name=Hadoop,
    description={Набор инструментов, предназначенный для обработки большого объема данных, используя сеть из нескольких компьютеров}
}
\newglossaryentry{hdfs}
{
    name=HDFS,
    description={Распределенная файловая система, используемая в \gls{hadoop}}
}
\newglossaryentry{webdav}
{
    name=WebDAV,
    description={Web Distributed Authoring and Versioning -- набор расширений протокола HTTP для совместного редактирования файлов на удаленных серверах}
}
